\documentclass[10pt]{article}
\newcommand{\RomanNumeralCaps}[1]
    {\MakeUppercase{\romannumeral #1}}
\usepackage{listings, xcolor,hyperref}
\setlength{\parskip}{1em}
\title{TDA342\\ Report for Replay Monad - Part \RomanNumeralCaps 1}
\author{QUFEI WANG \\ JACOB J. NILSSON}

\lstset{
breaklines=true,
frame=shadowbox,
language=Haskell
}

\begin{document}
\maketitle
\section*{Task 1}
File $Replay.hs$ includes the implementation details of the \textit{Replay} monad. The most important data type in this file is the 'replay monad' transformer, $ReplayT$, which is defined as a function. It accepts two $Traces$, where the first represents the accumulated $Results$ and $Answers$ and the second represents the unconsumed input $Results$ or $Answers$. The result of the function is a value of type $Either$ in the underlying monad. It is either a question with the currently accumulated answers and results, or a computation result with two traces that are prepared for further operations. 
\begin{lstlisting}
newtype ReplayT m q r a = ReplayT {replayT :: Trace r -> Trace r -> m (Either (q, Trace r) (a, Trace r, Trace r))}
\end{lstlisting}

Type constructor $'ReplayT\; m\; q\; r'$ is a monad, the basic functions $return$, $(>>=)$ are defined as follows:
\begin{lstlisting}
return' :: Monad m => a -> ReplayT m q r a                                                                                                                                                    
return' a = ReplayT $ \before after -> return $ Right (a, before, after)

bind :: Monad m => ReplayT m q r a -> (a -> ReplayT m q r b) -> ReplayT m q r b                                                                                                               
bind r1 f = ReplayT $ \before after -> do                                                                                                                                                     
  a1 <- replayT r1 before after                                                                                                                                                               
  case a1 of                                                                                                                                                                                  
    Left l -> return (Left l)                                                                                                                                                                 
    Right (a, t1, t2) -> replayT (f a) t1 t2
\end{lstlisting} 

$return'$ takes a value and makes it the result of the computation, leaving the input traces untouched. $bind$ first runs the monad represented by its first parameter, and inspects the result. If the result is $Left\; l$, which means a question is raised without an answer, then the result is returned for the whole computation, ignoring the function that follows. If the result is $Right(a,\; t1,\; t2)$, then traces $t1$ and $t2$ are used as the input parameters to evaluate the function returned by $f\; a$.

The generalized version of the two important constructors $io,\; ask$ are defined as follows:
 \begin{lstlisting}
iot :: (Show a, Read a, Monad m) => m a -> ReplayT m q r a                                                                                                                                    
iot m = ReplayT $ \before after -> case after of                                                                                                                                              
   [] -> do a <- m                                                                                                                                          
            return $ Right (a, before ++ [Result (show a)], [])                                                                                             
   (x:xs) -> case x of                                                                                                                                      
       Result s -> return $ Right (read s, before ++ [x], xs)                                                                                       
       _  -> error "type mismatch, expect: Result String, actual: Answer r"
                                                 
                                                 
askt :: Monad m => q -> ReplayT m q r r                                                                                                                                                       
askt q = ReplayT $ \before after -> do                                                                                                                                                        
   case after of                                                                                                                                                                              
    [] -> return $ Left (q, before)                                                                                                                                                           
    (x:xs) -> case x of                                                                                                                                                                       
                Answer r -> return $ Right (r, before ++ [x], xs)                                                                                                                             
                _        -> error "type mismatch, expect: Answer r, actual: Result String"
\end{lstlisting}

Both of $iot,\; askt$ pattern match on the second parameter $after$, which represents the upcoming $Results$ or $Answers$ that have not  been consumed. For $iot$, if $after$ is an empty trace, then it performs the monad $m$, gets its result $a$, and makes it the part of the result of the computation. It also appends $Result\; (show\; a)$ to the first parameter $before$, so that the following $ReplayT$ computations (if any) could use it as an updated accumulated traces. If $after$ is of form $(x:xs)$, then a pattern match is made on $x$: If $x$ is $Result\; s$, then it is converted to the target type by $read\; s$ and returned. Otherwise there's a mismatch between the expected value and the actual value, an error occurs. The same kind of reasoning can be applied on method $askt$, which we will not elaborate here.

What the generalized version of function $run$, which is $runt$, does is simply invoking the function of $ReplayT$ with $emptyTrace$ as the first paramter, and the acutal input as the second paramter. When the result is a $Left$, it is returned, otherwise it is curtailed to the appropriate type and returned.
\begin{lstlisting}
runt :: Monad m => ReplayT m q r a -> Trace r -> m (Either (q, Trace r) a)                                                                                                                    
runt rt t = let m = replayT rt emptyTrace t in                                                                                                                                                
  do a1 <- m                                                                                                                                                                                  
     case a1 of                                                                                                                                                                               
       Left l -> return $ Left l                                                                                                                                                              
       Right (a2, t1, t2) -> return $ Right a2
\end{lstlisting}

Other types and functions in $Replay.hs$ are either common auxiliary or derived from the functions above, therefore we will not give them explanations here, the commnets in the source file should be sufficient.

\section*{Task 2}
\begin{quote}                                                                                                                                                                                 
\textbf{Question: Why is it not possible to make your transformer an instance of MonadTrans ?}                                                                                      
\end{quote}
A data type which is an instance of $MonadTrans$ should have the type of kind $(* \to *) \to * \to *$, which can be seen as accepting a $monad$ and a concrete type to return a concrete type. Because the type parameters in the definition of our transformer is a $monad$ type $m$ followed by three concrete types $q$, $r$ and $a$, it is impossible to make our transformer an instance of $MonadTrans$. However, one can verify that our transformer behaves exactly like a $'MonadTrans'$ given function $liftR$ as the implementation of $lift$ by safisfying the following laws:
\begin{itemize}
\item $lift\; .\; return = return$
\item $lift\; (m >>= f) = lift\; m  >>=\;  (lift\; .\; f)$
\end{itemize}
\section*{Task 3}
We included four test suites into our library, which can be found under the directory $test$. Case '$test-replay-io$' is just the original test file downloaded from Canvas. Case '$test-replay-state$' uses similiar testing code as $test-replay-io$, the difference is that it replaces the $IO$ monad with a $State$ monad that is obtained by applying the type constructor $State$ (from $Control.Monad.State.Lazy$) on $Int$. The inner state is coded to record the number of invocations of method $iot$, so that the semantics of function $runProgram$ remains unchanged, even though its signature has changed to a non-monadic style $Program \to Input \to Result$. 

Case '$test-replay-monadLaws$' just uses the same  testing frame from '$test-replay-io$' to verify that $ReplayT$ abides by monad laws. Case '$test-replay-quickcheck$' uses $QuickCheck$ to generate random test cases to verify the monad laws. It does this by producing random values of needed types using custom generator, and verifying the monad laws as properties. 
\end{document}
