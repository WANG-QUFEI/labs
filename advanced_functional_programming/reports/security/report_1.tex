\documentclass[10pt]{article}
\newcommand{\RomanNumeralCaps}[1]
    {\MakeUppercase{\romannumeral #1}}
\usepackage{listings, xcolor,hyperref,amsmath}
\setlength{\parskip}{1em}
\title{TDA342\\ Security - Part \RomanNumeralCaps 1}
\author{QUFEI WANG \\ JACOB J. NILSSON}

\lstset{
breaklines=true,
frame=shadowbox,
language=Haskell
}

\begin{document}
\maketitle
\section*{Task 1}
The definition of data type $Sensitivity$ as a monoid is as follows:
\begin{lstlisting}
data Sensitivity = Secret | Public deriving (Show, Eq)

instance Monoid Sensitivity where
  mempty = Public
  mappend Public Public = Public
  mappend _ _ = Secret

instance Semigroup Sensitivity where
  (<>) = mappend
\end{lstlisting}
From this definition we have the following two premises:
\begin{enumerate}
\item $mempty = Public$
\item Only $Public$ $`mappend`$ $Public$ yields $Public$, in all other situations we have $Secret$ as the reuslt of $mappend$ operation.
\end{enumerate}
From these two premises, we have the following 4 conclustions.
\begin{enumerate}
\item $mappend\; mempty\; a == a$.

\textit{Proof.} From premise $2$ we have,
\begin{align} 
mappend\; Public\; Public = Public \\ 
mappend\; Public\; Secret = Secret
\end{align}
From premise $1$, we have
\begin{align} 
Public = mempty
\end{align}
Substitute $Public$ in $(1),(2)$ with $mempty$, we have
\begin{align} 
mappend\; mempty\; Public = Public \\ 
mappend\; mempty\; Secret = Secret
\end{align}
Then we conclude that $mappend\; mempty\; a == a$.

\item $mappend\; a\; mempty == a$.

\textit{Proof.} Similar with the proof above, just replace formula (2) with
$$mappend\; Secret\; Public = Secret$$, and apply the same logic to the rest.

\item $mappend\; a\; (mappend\; b\; c) == mappend\; (mappend\; a\; b)\; c$.

\textit{Proof.} Assume $a = mempty$, by the first conclusion, we have
\begin{align}
mappend\; a\; (mappend\; b\; c) = mappend\; b\; c \\
mappend\; (mappend\; a\; b)\; c = mappend\; b\; c
\end{align}, such that
\begin{multline}
a = mempty \implies \\ mappend\; a\; (mappend\; b\; c) == mappend\; (mappend\; a\; b)\; c
\end{multline}. Similarly, by assuming $b = mempty$ and $c = mempty$, using the first two conclusions we already proved, we have
\begin{multline}
b = mempty \implies \\ mappend\; a\; (mappend\; b\; c) == mappend\; (mappend\; a\; b)\; c
\end{multline}
\begin{multline}
c = mempty \implies \\ mappend\; a\; (mappend\; b\; c) == mappend\; (mappend\; a\; b)\; c
\end{multline}
Combining $(8),(9),(10)$, we know that the equation holds if any of $a, b, c$ is $mempty$. On the other hand, if all of $a, b, c$ are $Secret$, then by premise $(2)$, we have both left and right hand side of the equation is $Secret$. So the equation is proven.

\item $mappend\; a\; b == Public \iff (a == Public\; \&\&\; b == Public)$. This is a direct conclusion of premise (2).
\end{enumerate}
By the 4 conclusions we've proven above, we know that the data type $Sensitivity$ is a monoid.

\section*{Task 2}
The modifications made to achieve the desired effect on file $Basic.hs$ is listed as follows:
\begin{lstlisting}
data Expr = Lit Integer
		...
          | Sec Expr
  deriving (Show)

type Env = Map Name (Labeled Value)

lookupVar :: Name -> Eval (Labeled Value)

language = P.Lang
  { ...
  , P.lNewSecretexp = Sec
  , ...
  }
  
eval :: Expr -> Eval (Labeled Value)
eval (Lit n)       = return (LV Public n)
eval (a :+: b)     = do
  LV s1 v1 <- eval a
  LV s2 v2 <- eval b
  return $ LV (s1 `mappend` s2) (v1 + v2)
eval (Var x)       = lookupVar x
eval (Let n e1 e2) = do v <- eval e1
                        localScope n v (eval e2)
eval (Sec e)       = do LV _ v <- eval e
                        return $ LV Secret v
\end{lstlisting}
\begin{enumerate}
\item We add a value constructor $Sec\; Expr$ in type $Expr$ to incorporate the 'secret' keyword.
\item We changed the type of $Env$ to $Map\; Name\; (Labeled\; Value)$, since we need to track the sensitivity level of our expression.
\item As a consequence of 2, we also need to change the signature of function $lookupVar$. However the implementation of this function doesn't need to change since the definition of $Eval$ remains the same.
\item We modified the value of $P.lNewSecretexp$ in $language$ to $Sec$ to guarantee the parser works correctly.
\item Finally, we changed the signature of function $eval$ and altered parts of the implementation to have the final result being 'labeled'. Notice that we properly used the function $mappend$ to make the semantics  of sensitivity consistent.
\end{enumerate}
A testing function named $testLabeledValue$ is added to the source file to verify the correctness of our implementation. 
\end{document}