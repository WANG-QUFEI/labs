\documentclass[10pt]{article}
\setlength{\parskip}{1em}
\usepackage{amsmath,amsthm,mathtools,commath,xcolor,algorithmic,algorithm}
\renewcommand{\algorithmicrequire}{\textbf{Initialization:}}
\title{TDA251\\Solutions for Assignment 2}
\author{Wang QuFei\\900212-6952\\qufei@student.chalmers.se}

\begin{document}
\maketitle
\section*{Excrcise 3}
Let $c$ denote the minimum total weight of set $F \subset E$ such that the removal of $F$ disconnects $s$ and $t$, let $c'$ denote the capacity of a minimum $s,t$ cut $(A^{*},B^{*})$ in $G'$. We prove that $c = c'$ and the edges connect $A^*$ and $B^*$ constitute exactly the solution $F$ of $G$.
\begin{itemize}
	\item Fist, for the flow $f$ that gives the minimum cut $(A^{*},B^{*})$, we claim that for all opposite directed edges $e=(u,v)$ and $e'=(v,u)$, f can be modified such that either $f(e)=0$ or $f(e')=0$. For if $f(e) \neq 0$, $f(e') \neq 0$, let $\delta$ be the smaller of the two values, and modify $f$ by decreasing the flow value on both $e$ and $e'$ by $\delta$. The resulting flow $f'$ is feasible, has the same value as $f$, and the capacity condition and conservation condition on vertex $u$ and $v$ still hold. We thus assume that for all opposite directed edges, the flow value of one of them must be zero under $f$.
	\item Second, for all undirected edges that connect $A^*$ and $B^*$ in $G$ with a weight $w > 0$, we claim that flow $f$ in $G'$ gives directed edge $e=(u,v)$ flow value $f(e)=w$, gives directed edge $e'=(v,u)$ flow value $f(e')=0$, where $u \in A^*$, $v \in B^*$. For otherwise there would be a $s,v$ path in the residual graph of $G'$ which contradicts our assumption that $v \in B^*$.
	\item We say $c \leq c'$. For from
		\[c' = \sum_{\substack{e\, out\, of\, A^*\\ e \in G'}}f(e) = \sum_{\substack{e\, connects\, A^*,B^*\\ e\, \in G}}w(e)\]
we know that $c'$ is the total weight of a set of edges in $G$, with the removal of which we can disconnect $s$ and $t$ in $G$, whereas $c$ is by definition the total weight of edges in $G$ with the same property with minimum value.
	\item We say $c' \leq c$. Assume $F \subset E$ is the set of edges that yields minimum total weight $c$. Denote $A'$ as the set of nodes reachable from $s$ in $G$ after the removal of $F$, $B' = V \setminus A'$. Clearly, $(A',B')$ forms a cut in $G'$, and for each of these edges $e=(u,v) \in F, u \in A', v \in B'$, we assign both directed edges $e_1=(u,v), e_2=(v,u)$ in $G'$ with flow value $w(e)$, assign other edges in $G'$ with flow value 0. This satisfies capacity condition and conservation condition for every node in $G'$, and the capacity of cut $(A',B')$
		\[c(A',B') = \sum_{\substack{e\, out\, of\, A'\\ e \in G'}}f(e) = \sum_{\substack{e\, connects\, A',B'\\ e\, \in G}}w(e) = c\]
Because $c'$ is by definition the value of minimum cut of $G'$, so we have $c' \leq c(A',B') = c$.
\end{itemize}
\qed
\section*{Excrcise 4}
\textbf{4.1} Denote the size of $F$ as $|F| = n$. Considering only the free squares in $F$, we now reduce this problem to Maximum Flow problem as follows.
\begin{itemize}
	\item Each square represents a node.
	\item Add two nodes $s$ and $t$ as the source and the sink.
	\item Add directed edges from $s$ to each white square with capacity 1.
	\item Add directed edges from each white square to all of its neighboured(N-S or W-E direction) black squares with capacity 1.
	\item Add directed edges from each black square to node $t$ with capacity 1.
	\item Denote $G=(V, E)$ as the directed flow network with the nodes and edges stated above. 
\end{itemize}
We claim that a value $k$ of maximum flow in $G$ exists if and only if an optimal solution of $k$ boxes exists in the storage hall problem.
\begin{itemize}
	\item Let $k$ and $k'$ respectively be an optimal solution to the maximum flow problem and the storage hall problem.
	\item Each edge from graph $G$ has capacity 1, a flow of value $k$ means $k$ pairwise disjoint paths from node $s$ to node $t$. Each path covers a pair of nodes, which can be seen as a box placed on this pair of nodes. Thus we have a solution of $k$ boxes for the storage hall problem. Since $k'$ is the optimal value, we have $k \leq k'$.
	\item For the $k'$ boxes that placed on the set $F$ of squares, each box can be viewd as an edge connecting a white node and a black node with flow of value 1. Since two boxes can not overlap, we have $k'$ pairwise disjoint paths from node $s$ to node $t$ with a flow of value $k'$. Since $k$ is the maximum value of flow, we have $k' \leq k$.
	\item We proved that $k = k'$
\end{itemize}
Running \emph{Ford-Fulkerson Algorithm} in $G$ costs $O(mC)$ time, where $m$ represents the number of edges and $C$ represents sum of the capacity of the edges leaving $s$. For the set $F$ with $n$ free squares, $m$ has size $O(n)$, C has size $O(n)$, thus time bound of this problem would be $O(n^2)$\qed \\
\textbf{4.2} {\color{red}Suppose $f$ is a max-flow we get from 4.1 by running the \emph{Ford-Fulkerson Algorithm}. Denote $G_f$ as the residual graph with respect to $f$, we know that there's no s-t path in $G_f$.\\
Now, one square is added to $F$. Add node $e_n$ in $G_f$ as the representation of this newly freed square. By the way of reduction in 4.1, if this new square is white, we need to add one directed edge from $s$ to $e_n$, and at most 4 directed edges from $e_n$ to the nodes which represent the neighboured black squares. Otherwise, we need to add one directed edge from $e_n$ to $t$, and at most 4 directed edges from the nodes which represent the neighboured white squares to $e_n$. The capacity of each of these newly added edges is 1, and the flow value on these edges is 0.\\
Denote the graph with this new node and edges as $G'$. A more efficient method can be given as follows, it searches all s-t paths in the $G'$, the number of these possible s-t pahts in $G'$ is the extra flow of value that can be added, which is also the number of extra boxes that be added for the original problem.
}
\begin{algorithmic}
\STATE \textbf{Initialization:}\\
\begin{itemize}
	\item A boolean value $complete$ signifies whether a  s-t path $p_{s,t}$ has been found, initialize it with 0.
	\item Initialize stack $S$ with $e_n$ as its top element. 
	\item A boolean value $direction$, when set 1, means we are constructing $p_{s,t}$ forwardly from $s$ to $t$, otherwise, we are constructing $p_{s,t}$ backwardly from $t$ to $s$.
	\item If $e_n$ is white, set $direction$ to 1, since we already have a partial s-t path $s \to e_n \to \dots$, what we need next is to complete the construction of this path forwardly. Mark $s$ and $e_n$ as \emph{explored}.
	\item If $e_n$ is black, set $direction$ to 0, since we already have a partial s-t path $\dots \to e_n \to t$, what we need next is to complete the construction of this path backwardly. Mark $t$ and $e_n$ as \emph{explored}.
\end{itemize}
\WHILE{$S$ not empty \AND $complete == 0$}
\STATE $e =  pop(S)$;\\
\IF{$direction == 1$}
	\IF{$e$ == $t$}
	\STATE assign $complete = 1$;
	\ELSE
		\FORALL{ unexplored nodes $n$ in $G'$ for which $(e, n)$ is a directed edge}
		\STATE mark $n$ as \emph{explored};\\ push $n$ to $S$;
		\ENDFOR
	\ENDIF
\ELSE
	\IF{$e$ == $s$}
	\STATE assign $complete = 1$;
	\ELSE
		\FORALL{ unexplored nodes $n$ in $G'$ for which $(n, e)$ is a directed edge}
		\STATE mark $n$ as \emph{explored};\\ push $n$ to $S$;
		\ENDFOR
	\ENDIF
\ENDIF
\ENDWHILE
\RETURN $complete$
 \end{algorithmic}
If this method returns 1, then we conclude that a better solution exists and one more box can be placed. Otherwise there isn't a better solution. Since each node is examined only once, with some constant time searching for its neighbours when using adjacency list as the representation of $G'$, the time bound of this method is $O(n)$.\\
{\color{red}
Next we prove that at most one more box can be inserted. The procedure described above ends when it finds one s-t path, which can be denoted as $\delta$. By setting the flow value on all the edges in $\delta$ to 1, we get a flow $f'$ which yields one more value than $f$, which means we can at least add one more box for the original problem. \\
Suppose there are situations where more than one boxes can be inserted. Because the capacity of all the edges in $G'$ is 1, this means that there must be another s-t path $\delta'$. We claim that $\delta'$ can not pass $e_n$, otherwise if $e_n$ represents a white square, there's only one edge from $s$ to $e_n$; If $e_n$ represents a black square, there's only one edge from $e_n$ to $t$. On both cases it would mean $\delta$ and $\delta'$ share an edge from $s$ or into $t$, which is impossible due to the fact that the capacity of each edge is 1.This means $\delta'$ is indenpent of $e_n$, which means $\delta'$ exists in $G_f$. This contradicts to the fact that there's no s-t path in $G_f$. Thus we have proved that at most one box can be added if a better solution exists.\\
}
\qed
\end{document}